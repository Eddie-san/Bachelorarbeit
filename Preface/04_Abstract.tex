% !TeX root = ../thesis.tex


\chapter*{Kurzfassung}
\label{sec:abstract}
\addcontentsline{toc}{chapter}{Kurzfassung}

Das Ziel dieser Arbeit ist es, einen \gls{acr:VLC}-Sender mit einer automatischen Amplitudenregelung und variablem Offset mithilfe eines Arduino Mikrocontrollers zu realisieren. Anschließend war es das Ziel mit dem Aufbau über den physikalischen Kanal, das Licht, eine Übertragungsstrecke aufzubauen. Dazu wurde der von Herrn Prof. Dr. Litzenburger aufgebaute \gls{acr:VLC}-Empfänger als Gegenstück dieses entwickelten Senders verwendet.

Als Übertragungsmedium wurden Audiosignale gewählt, welche mithilfe einer vom Fraunhofer Institut konzipierten Software namens Dream in die gewünschte Übertragungsform umgewandelt wurden. Dabei handelt es sich um ein \gls{acr:DRM}-Signal, welches die \gls{acr:OFDM} Modulationsform verwendet.

Entsprechend wurden sowohl geeignete Bauteile für die Hardwarerealisierung, als auch eine kompatible Software zur Simulation und Projektierung dieser Anforderungen ausgewählt, implementiert, und getestet, um die oben aufgeführten Funktionalitäten zu erzielen.