% !TeX root = ../thesis.tex
\chapter*{Acknowledgement / Danksagung}

\textit{\textbf{Die Neugier steht immer an erster Stelle eines Problems, das gelöst
		werden will.}}
\textit{— Galileo Galilei}


Im Studiengang Elektro- und Informationstechnik an der Hochschule Karlsruhe – Technik und Wirtschaft in Karlsruhe ist für die Beendigung des Bachelorstudiums eine Bachelor Thesis vorgesehen. Großes Interesse und Begeisterung an der elektrischen Schaltungstechnik bewegten mich zu meiner Entscheidung meine Abschlussarbeit im Bereich der \gls{acr:VLC} zu verfassen. Mir wurde die Möglichkeit zuteil, vier Monate im Labor der Hochschule Karlsruhe zu forschen, meinen Lösungsansatz zu realisieren und zu dokumentieren.

Hierbei konnte ich mein im Studium erlangtes Wissen anwenden, mir neue Fähigkeiten aneignen und auch umfangreiche Einblicke in die Elektrotechnik der aktuellsten kommunikationstechnischen Modelle bekommen. Deshalb möchte ich an dieser Stelle der Hochschule Karlsruhe für die produktive und intensive Zusammenarbeit danken. Mein besonderer Dank gilt Herrn Prof Dr. – Ing. Manfred Litzenburger, der mich während meiner Arbeit sehr engagiert betreut und stets sowohl konstruktiv, als auch hilfsbereit unterstützt hat.

Des Weiteren gilt mein Dank allen Professoren der Fakultät EIT, die mich im Vorfeld in sehr umfangreichen Vorlesungen und Veranstaltungen mit einem theoretischen sowie auch einem praktischen Basiswissen ideal auf meine Abschlussarbeit vorbereitet haben.

Ein herzliches Dankeschön gilt auch meiner Familie und Freunde, ohne deren Unterstützung
diese Arbeit nicht möglich gewesen wäre.

Zur Vereinheitlichung des Sprachgebrauches verwende ich in dieser Abschlussarbeit Abkürzungen. Ich verweise diesbezüglich auf das hier gegebene Abkürzungsverzeichnis.
