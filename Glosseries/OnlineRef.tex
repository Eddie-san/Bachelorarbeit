	% online references
\newglossaryentry{online:Eigen}{
	type=onlineref, 
	name={[01]}, 
	description={\hspace{1.5mm}Eigene Darstellung\newline 
		\href{}{}\newline 
	}}
\newglossaryentry{online:arduino}{
	type=onlineref, 
	name={[02]}, 
	description={\hspace{1.5mm}What is Arduino?\newline 	
		\href{https://www.arduino.cc/en/Guide/Introduction}{https://www.arduino.cc/en/Guide/Introduction}\newline
		Abgerufen 2021-02-09 }}
\newglossaryentry{online:spice}{
	type=onlineref, 
	name={[03]}, 
	description={\hspace{1.5mm}LT-Spice\newline 	
		\href{https://en.wikipedia.org/wiki/LTspice}{https://en.wikipedia.org/wiki/LTspice}\newline
		Abgerufen 2021-02-13}}
\newglossaryentry{online:dream}{
	type=onlineref, 
	name={[04]}, 
	description={\hspace{1.5mm}Dream\newline 	
		\href{http://www.mynetcologne.de/~nc-keilje/drm/dream/index.htm}{http://www.mynetcologne.de/~nc-keilje/drm/dream/index.htm}\newline
		Abgerufen 20201-03-10}}
\newglossaryentry{online:eagle1}{
	type=onlineref, 
	name={[05]}, 
	description={\hspace{1.5mm}Was ist Eagle?\newline 	
		\href{https://www.autodesk.de/}{https://www.autodesk.de/products/eagle/overview}\newline
		Abgerufen 2021-03-10}}	
\newglossaryentry{online:eagle2}{
	type=onlineref, 
	name={[06]}, 
	description={\hspace{1.5mm}Eagle\newline 	
		\href{https://de.wikipedia.org/wiki/Eagle\_(Software)}{https://de.wikipedia.org/wiki/Eagle\_(Software)}\newline
		Abgerufen 2021-03-10}}		
\newglossaryentry{online:mosfet}{
	type=onlineref, 
	name={[07]}, 
	description={\hspace{1.5mm}IGBT or MOSFET: Choose Wisely.\newline 	
		\href{https://www.infineon.com/dgdl/choosewisely.pdf?fileId=5546d462533600a40153574048b73edc}{https://www.infineon.com/dgdl/choosewisely.pdf?fileId=5546d462533600a40153574048b73edc}\newline
		Abgerufen 2021-03-15}}
\newglossaryentry{online:vtoi}{
	type=onlineref, 
	name={[08]}, 
	description={\hspace{1.5mm}Precision Voltage to Current Converter\newline 	
		\href{https://wiki.analog.com/university/courses/electronics/text/chapter-4}{https://wiki.analog.com/university/courses/electronics/text/chapter-4}\newline
		Abgerufen 2021-03-15}}
\newglossaryentry{online:leiterbahn}{
	type=onlineref, 
	name={[09]}, 
	description={\hspace{1.5mm}Leiterbahnen\newline 	
		\href{https://www.elektronikentwickler-aachen.de/layouterstellung/layouterstellung\_44.htm}{https://www.elektronikentwickler-aachen.de/layouterstellung/layouterstellung\_44.htm}\newline
		Abgerufen 2021-03-16}}
\newglossaryentry{online:thermo}{
	type=onlineref, 
	name={[10]}, 
	description={\hspace{1.5mm}Berechnung eines Kühlkörpers\newline 	
		\href{https://www.fischerelektronik.de/fileadmin/fischertemplates/download/Katalog/technischeerlaeuterungen\_d.pdf}{https://www.fischerelektronik.de/fileadmin/fischertemplates/download/Katalog/technischeerlaeuterungen\_d.pdf}\newline
		Abgerufen 2021-03-16}}
\newglossaryentry{online:thermo1}{
	type=onlineref, 
	name={[11]}, 
	description={\hspace{1.5mm}Wärmewiderstand beim Kühlkörper \newline 	
		\href{https://https://www.fischerelektronik.de/service/technische-informationen/kuehlkoerper-berechnen/}{https://www.fischerelektronik.de/service/technische-informationen/kuehlkoerper-berechnen/}\newline
		Abgerufen 2021-03-16}}
\newglossaryentry{online:thermo2}{
	type=onlineref, 
	name={[12]}, 
	description={\hspace{1.5mm}Galvanische Trennung \newline 	
		\href{https://www.cmc.de/page/thermal-management}{https://www.cmc.de/page/thermal-management/}\newline
		Abgerufen 2021-03-16}}
\newglossaryentry{online:elektronik}{
	type=onlineref, 
	name={[13]}, 
	description={\hspace{1.5mm}Vorlesungsskript Elektronik \newline 	
		\href{}{Prof. Dr.-Ing. Rudolf Koblitz - Hochschule Karlsruhe - Technik und Wirtschaft}\newline
	}}
\newglossaryentry{online:abtastung}{
	type=onlineref, 
	name={[14]}, 
	description={\hspace{1.5mm}Abtastfrequenz \newline 	
		\href{https://de.wikipedia.org/wiki/Abtastung\_(Signalverarbeitung)}{https://de.wikipedia.org/wiki/Abtastung\_(Signalverarbeitung)}\newline
		Abgerufen 2021-03-17}}
\newglossaryentry{online:elektromag}{
	type=onlineref, 
	name={[15]}, 
	description={\hspace{1.5mm}Elektromagnetische Welle \newline 	
		\href{https://de.wikipedia.org/wiki/Elektromagnetische\_Welle}{https://de.wikipedia.org/wiki/Elektromagnetische\_Welle}\newline
		Abgerufen 2021-03-18}}
\newglossaryentry{online:konst}{
	type=onlineref, 
	name={[16]}, 
	description={\hspace{1.5mm}Octave/Matlab - Communication System  \newline 	
		\href{http://www.sharetechnote.com/html/Octave\_Matlab\_Communication.html}{http://www.sharetechnote.com/html/Octave\_Matlab\_Communication.html}\newline
		Abgerufen 2021-03-19}}
\newglossaryentry{online:ham}{
	type=onlineref, 
	name={[17]}, 
	description={\hspace{1.5mm}Hamming Abstand  \newline 	
		\href{https://de.wikipedia.org/wiki/Hamming-Abstand}{https://de.wikipedia.org/wiki/Hamming-Abstand}\newline
		Abgerufen 2021-03-19}}
\newglossaryentry{online:quint}{
	type=onlineref, 
	name={[18]}, 
	description={\hspace{1.5mm}Vorlesungsskript Nachrichtentechnik  \newline 	
		\href{Prof. Dr.-Ing. Franz Quint - Hochschule Karlsruhe - Technik und Wirtschaft}{Prof. Dr.-Ing. Franz Quint
			 - Hochschule Karlsruhe - Technik und Wirtschaft}\newline}}
\newglossaryentry{online:bandbreite}{
	type=onlineref, 
	name={[19]}, 
	description={\hspace{1.5mm}Bandbreite \newline 	
		\href{https://de.wikipedia.org/wiki/Bandbreite}{https://de.wikipedia.org/wiki/Bandbreite}\newline
		Abgerufen 2021-03-20}} 
\newglossaryentry{online:drm}{
	type=onlineref, 
	name={[20]}, 
	description={\hspace{1.5mm}The Hitchhikers Guide to	Digital Radio Mondiale (DRM) \newline 	
		\href{https://www.drm-sender.de/?page=drm\&lang=en}{https://www.drm-sender.de/?page=drm\&lang=en}\newline
		Abgerufen 2021-03-21}}	
\newglossaryentry{online:drmpic}{
	type=onlineref, 
	name={[21]}, 
	description={\hspace{1.5mm}Digital Radio Mondiale \newline 	
		\href{https://de.wikipedia.org/wiki/Digital\_Radio\_Mondiale}{https://de.wikipedia.org/wiki/Digital\_Radio\_Mondiale}\newline
		Abgerufen 2021-03-21}}	
\newglossaryentry{online:aufbauarduino}{
	type=onlineref, 
	name={[22]}, 
	description={\hspace{1.5mm}Aufbau eines Arduino UNO \newline 	
		\href{https://www.grund-wissen.de/elektronik/arduino/aufbau.html}{https://www.grund-wissen.de/elektronik/arduino/aufbau.html}\newline
		Abgerufen 2021-03-15}}
\newglossaryentry{online:analogread}{
	type=onlineref, 
	name={[23]}, 
	description={\hspace{1.5mm}analogRead()\newline 	
		\href{https://www.arduino.cc/reference/de/language/functions/analog-io/analogread/}{https://www.arduino.cc/reference/de/language/functions/analog-io/analogread/}\newline
		Abgerufen 2021-03-12}}	
\newglossaryentry{online:fritz}{
	type=onlineref, 
	name={[24]}, 
	description={\hspace{1.5mm}Fritzing Beta Version\newline 	
		\href{https://fritzing.org/}{https://fritzing.org/}\newline
		Abgerufen 2021-03-12}}
\newglossaryentry{online:vac}{
	type=onlineref, 
	name={[25]}, 
	description={\hspace{1.5mm}Virtual Audio Cable\newline 	
		\href{https://virtual-audio-cable.de.uptodown.com/windows}{https://virtual-audio-cable.de.uptodown.com/windows}\newline
		Abgerufen 2021-03-24}}
\newglossaryentry{online:basis}{
	type=onlineref, 
	name={[26]}, 
	description={\hspace{1.5mm}Basisband\newline 	
		\href{https://de.wikipedia.org/wiki/Basisband}{https://de.wikipedia.org/wiki/Basisband}\newline
		Abgerufen 2021-03-29}} 
\newglossaryentry{online:ofdm}{
	type=onlineref, 
	name={[27]}, 
	description={\hspace{1.5mm}Orthogonal Frequency Division Multiplex\newline 	
		\href{https://de.wikipedia.org/wiki/Orthogonales\_Frequenzmultiplexverfahren}{https://de.wikipedia.org/wiki/Orthogonales\_Frequenzmultiplexverfahren}\newline
		Abgerufen 2021-03-29}}
\newglossaryentry{online:burst}{
	type=onlineref, 
	name={[28]}, 
	description={\hspace{1.5mm}Burstfehler\newline 	
		\href{https://de.wikipedia.org/wiki/Burstfehler}{https://de.wikipedia.org/wiki/Burstfehler}\newline
		Abgerufen 2021-03-29}}
\newglossaryentry{online:eagleleit}{
	type=onlineref, 
	name={[29]}, 
	description={\hspace{1.5mm}Layout\newline 	
		\href{https://www.elektronikentwickler-aachen.de/layouterstellung/layouterstellung\_43.htm }{https://www.elektronikentwickler-aachen.de/layouterstellung/layouterstellung\_43.htm}\newline
		Abgerufen 2021-03-29}}
\newglossaryentry{online:lcd}{
	type=onlineref, 
	name={[29]}, 
	description={\hspace{1.5mm}Liquid Crystal Display\newline 	
		\href{https://create.arduino.cc/projecthub/Arnov\_Sharma\_makes/lcd-i2c-tutorial-664e5a}{https://create.arduino.cc/projecthub/Arnov\_Sharma\_makes/lcd-i2c-tutorial-664e5a}\newline
		Abgerufen 2021-04-05}}
\newglossaryentry{online:I2C}{
	type=onlineref, 
	name={[29]}, 
	description={\hspace{1.5mm}I2C\newline 	
		\href{https://de.wikipedia.org/wiki/I²C}{https://de.wikipedia.org/wiki/I²C}\newline
		Abgerufen 2021-04-05}}