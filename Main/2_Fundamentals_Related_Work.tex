% !TeX root = ../thesis.tex

\chapter{Grundlagen}
\label{chap:fundamentals_related-work}
Hier steht die Einleitung in die Grundlagen.

\section{Abschnitt 1}
\label{sec:Abschnitt1}

Hier steht der erste Abschnitt.


\subsection{Unterabschnitt 1.1}
\label{subsec:Unterabschnitt1}

Hier steht der erste Unterabschnitt.


\subsubsection{Unterunterabschnitt 1.1.1}
\label{subsub:Unterunterabschnitt1}

Hier steht der erste Unterunterabschnitt

\subsection{Unterabschnitt 1.2}
\label{subsec:Unterabschnitt12}

Hier steht der zweite Unterabschnitt.


\section{Beispiele}
\label{sec:Bsp}


Hier sind einige Beispiele.

\subsection{Abkürzung}
\label{subsec:abkürzung}

So verwende ich eine Abkürzung \gls{acr:djt} und so erneut \gls{acr:djt}.

\subsection{Symbol}
\label{subsec:symbol}
So füge ich ein Symbol ein \gls{symb:h_s}.

\subsection{Bild}
\label{subsec:Bild}
So füge ich ein Bild ein:

\begin{figure}[H]
	\centering
	\includegraphics[width = 0.7 \textwidth ]{donald.jpg}
	\caption[Titel für das Abbildungsverzeichnis]{Bild Unterschrift}
	\label{fig:donald}
\end{figure}

So beziehe ich mich auf das Bild ~\ref{fig:donald}

\subsection{Formel}
\label{subsec:formel}

\begin{equation}
	\label{equ:bsp1}
	F_{A} \geq 2 \cdot F_{S}
\end{equation}

So beziehe ich mich auf die Formel ~\ref{equ:bsp1}


\subsection{Tabelle}
\label{subsec:table}
\begin{table}[htb]
	\begin{center}
		\begin{tabular}[h]{|c|c|c|}	
			\hline
			Stimmen für SleepyJoe & davon gefaked  & Sieger \\
			\hline
			80000000 & 80000000 & Donald \\
			\hline
		\end{tabular}
		\caption{Unterschrift  der Tabelle}
		\label{tab:Tabelle1}
	\end{center}
\end{table}

So beziehe ich mich auf die Tabelle ~\ref{tab:Tabelle1}


\subsection{Referenz und Zitat}
\label{subsec:refcite}
Referenzierung auf ein Kapitel ~\ref{sec:Abschnitt1}.
Zitat einfügen
\cite{donaldtrumpDonaldWilde2020}.

So verwende ich eine Online Quelle \gls{online:donald}
