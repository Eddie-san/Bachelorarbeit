% !TeX root = ../thesis.tex

\chapter{Fazit}
\label{sec:conclusion_future-work}

Zusammenfassend konnte gezeigt werden, dass im Zuge dieser Abschlussarbeit ein funktionierender \gls{acr:VLC}-Sender aufgebaut werden konnte. Die in der Einleitung im Kapitel~\ref{sec:The aim of the work} festgelegten Rahmenbedingungen konnten ausnahmslos eingehalten und in einem Aufbau umgesetzt werden. Zudem wurden noch einige Zusatzleistungen wie das Status \gls{acr:LCD}, das 3D-Druck-Gehäuse und der Lüfter zur Anregung des Luftstroms im Gehäuse, mit eingearbeitet werden um die Funktionen des \gls{acr:VLC}-Senders zu erweitern. 

Bei der Implementierung der Amplitudenregelung wurde jedoch klar, dass der Arduino sich zum auslesen des Sendesignals nicht optimal eignet. Aufgrund seiner zu niedrigen Abtastfrequenz von 15kHz kann somit das Signal von 20kHz nicht vollständig rekonstruiert werden. Der Arduino liefert über die im Code ausgelesene Fensterfunktion in dieser Abschlussarbeit jedoch ein ausreichend quantitatives Ergebnis, mit welchem eine Amplitudenregelung zu genüge durchgeführt werden kann. Eine gute Alternative um diesen Schritt zu optimieren, liefert beispielsweise ein \gls{acr:PSoC}-Mikrocontroller. Dieser besitzt einen Delta-Sigma-\gls{acr:AD}-Wandler, welcher sich gut für messtechnische Anwendungen eignet. Der \gls{acr:PSoC} 5LP z.B. enthält einen solchen Delta-Sigma-\gls{acr:AD}-Wandler, welcher eine maximale Abtastrate von bis zu 384kHz bei einer 8 Bit Auflösung besitzt. Somit könnte der Mikrocontroller den Signalverlauf genauer abtasten, besser rekonstruieren und die Amplitudenregelung dementsprechend weiter präzisiert werden. 

Des weiteren limitiert das digitale Potentiometer in der Rückkopplung der ersten analogen Signalverarbeitungsstufe, durch seine obere Spannungsgrenze von 6V die maximale Amplitudenverstärkung. Dies hat zur Folge, dass sehr schwache Signale durch die Amplitudenregelung unter Umständen nicht stark genug verstärkt werden können.

Als letzten Punkt lässt sich jedoch sagen, dass mithilfe der Recherche und Ausarbeitung im Rahmen der Abschlussarbeit, nicht nur audiospezifische, sondern grundlegende Funktionsweisen fortschrittlicher nachrichtentechnischer Übertragungssysteme verdeutlicht und unter technischen Standpunkten kritisch evaluieren werden konnten.